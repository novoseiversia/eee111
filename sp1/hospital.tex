\documentclass{article}



\title{EEE111 SP1 Documentation}
\author{Nile Jocson \textless{novoseiversia@gmail.com}\textgreater}
\date{October 11, 2024}



\begin{document}
	\maketitle
		\pagebreak



	\tableofcontents
		\pagebreak



	\section{Overview}
		The program is a CLI supply and inventory monitoring program based on
		the EEE111 SP1 specifications provided. It accepts the following commands:

		\begin{itemize}
			\item \verb|<file_name:str> needed_now| --- prints out the amount of items
			needed to fulfill the item shortage for the current day.
			\item \verb|<file_name:str needed_in <X:int>| --- prints out the amount of items
			needed to fulfill the item shortage for the following \verb|X| days.
			\item \verb|<file_name:str> runs_out| --- prints out the first item to run out, and
			in how many days it will.
			\item \verb|<file_name:str> run_outs| --- prints out the first \verb|N| items to run
			out, and in how many days they will.
			\item \verb|help| prints out the help text.
			\item \verb|exit| exits the program.
		\end{itemize}

		The program will accept commands indefinitely, exiting when the \verb|exit| command is
		entered. If an invalid command is entered, the program prints out the help text.

		\pagebreak

	\section{Parsing}
		Parsing utilities were created in order to simplify parsing inputs. This eliminates the
		need for long if-elif-else chains and complicatede string comparisons, and instead
		abstracts them into an easy-to-use API\@. This API consists of two dataclasses and
		two functions:

		\begin{verbatim}
			@dataclass
			class TransformInfo:
				convert : type | Callable
				position: int

			@dataclass
			class ParseRule:
				transforms : list[TransformInfo]
				find_string: str | None = None

			def parse_rules(
				rules: list[ParseRule],
				args: list[str]
			) -> list[Any] | None

			def parse_rulesets(
				rulesets: list[list[ParseRule]],
				args: list[str],
				default: list[Any]
			) -> list[Any]:
		\end{verbatim}

		The logic of this API can be broken down into three processes.

		\subsection{Input}
			\subsubsection{Arguments}
				The parsing API uses input in the form of \verb|list[str]|. It does not handle
				actual program IO, nor does it handle the splitting of the input lines. This
				input type is derived from the return type of \verb|str.split()|, which was
				used to split user input by a specified delimiter. From now on, the term
				``arguments'' will be used to refer to user input with type \verb|list[str]|
				given to the parsing API\@.

			\subsubsection{Rules}
				A rule determines if an argument is valid. It specifies a transform which contains
				either a type that the argument must be convertible to, or a function of which the
				argument must be a valid parameter of. In addition, it may also specify a string
				which the argument must be equal to; this is used to disambiguate between command
				names. If any one of these were not satisfied, the argument is invalid. A rule is
				defined with the \verb|ParseRule| class.

				Most programs however, take in multiple arguments, thus requiring multiple rules.
				Rules are simply a \verb|list[ParseRule]|, where each element has a corresponding
				argument (corresponding rules and elements have the same index). Each element is
				validated against its corresponding rule. If any one rule is not satisfied, the
				whole argument list is invalid.

				Here are some examples of rules, and their valid and invalid arguments:

				\begin{center}
					\begin{tabular}{c c c}
						Rules & Valid & Invalid \\
						\\
						{[int]} & {[5]} & {[``a'']} \\
						{[int, int]} & {[5, 10]} & {[5, ``a'']} \\
						{[``add'', int, int]} & {[``add'', 5, 10]} & {[``sub'', ``5'', ``10'']} \\
						{[``abs'', int]} & {[``abs'', -5.4]} & {[``absv'', -5.4]} \\
					\end{tabular}
				\end{center}

				If the arguments given to \verb|parse_rules| are invalid, it returne \verb|None|.
				If not, it moves on to the next process.

\end{document}
